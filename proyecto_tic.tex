\documentclass[a4paper,12pt]{article}
\usepackage[utf8]{inputenc}
\usepackage[spanish]{babel}
\usepackage{parskip}
\usepackage{amsmath}
\usepackage{graphicx}
\usepackage{hyperref}
\title{Desarrollo de Proyecto}
\author{Ana Portocarrero}
\date{30 de julio de 2025}


\begin{document}

\maketitle

\subsection*{Objetivo}

Crear un script en el cual se dessarrolle un juego interactivo de Tres en Línea (también conocido como “Gato” o “Tic Tac Toe”) utilizando el lenguaje de programación Python y ejecutando desde la consola. que permita la participación de dos jugadores por turnos, que permita la interacción a través de la terminal y que sea capaz de validar las entradas y si se trata de una victoria o un empate; en un periodo de una semana  con la finalidad de aplicar y reforzar las habilidades de programación que tengo y aprender más sobre el desarrollo de juegos.


\subsubsection*{\textit{Paso 2}}
Elegí programar ej juego clásico  “gato” porque es un juego simple pero interesante en el que puedo aplicar habilidades de programación como funciones, ciclos, manejar errores y validación de entradas y es un buen punto de partida para aprender a diseñar el flujo, aplicar estructuras e identificar los patrones de este tipo de juegos 


Asímismo, me permite avanzar a proyectos más complejos, además de que el jugador desarrolla habilidades de estrategia y razonamiento.

 

\subsubsection*{Paso 3}

\begin{enumerate}

  \item \textbf{¿Qué quieres lograr con tu proyecto?} \\

  Quiero desarrollar un juego funcional de Gato en Python que permita la interacción por turnos entre dos jugadores  desde la consola, para poder resolver otros proyectos más complejos a futuro 



  \item \textbf{¿Para quién estás programando?} \\ 
  Para mi propio proceso de aprendizaje y palguna persona que quiera jugar en la consola sin la necesidad de experiencia previa, solo al alcance de una computadora. El usuario final serán dos jugadores que interactuan desde el teclado, incluso yo misma como desarrolladora

  \item \textbf{¿Qué decisiones necesita tomar el programa?} \\ 
  Validar entradas del usuario, cambiar de turnos, detectar victorias o empates y finalizar el juego adecuadamente, y poder iniciar nuevas partidas

  \item \textbf{¿Cómo se puede romper el programa y cómo prevenirlo?} \\ 
  El programa puede fallar si se ingresan datos inválidos o se intenta jugar en casillas ocupadas. Se previene validando los datos con funciones y verificando que las casillas estén libres.

  \item \textbf{¿Cómo prevenir sesgos?} \\ 
  El juego es inclusivo y neutral, no solicita nombres, géneros ni características personales, no se almacenan datos, y lo puede jugar una persona cualquiera al alcance de una terminal, 

\end{enumerate}
\end{document}
